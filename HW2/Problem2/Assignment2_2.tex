\documentclass[11pt]{article}
\usepackage{fullpage,amsthm,amsfonts,amssymb,epsfig,amsmath,times,amsthm}
\usepackage{tabu} 

\newtheorem{theorem}{Theorem}
\newtheorem{claim}[theorem]{Claim}
\newcommand\tab{\setlength\parindent{24pt}}

\begin{document}
	
	\begin{center}
		{\bf\Large CMPS 102 --- Fall 2018 --  Homework 2}\\
		Alyssa Melton\\
		I have read and agree to the collaboration policy. \\
		Collaborators: none\\
	\end{center}
	
	%------------------------------------------------------------------------------	
	%------------------------------------------------------------------------------
	
	\section*{Solution to Problem 2}
	Important definitions from the problem:\\
	$A[1] \leq A[2]$ \\
	$A[n-1] \geq A[n]$\\
	An element $A[x]$ is a peak if $A[x] \leq A[x-1]$ and $A[x] \geq A[x+1]$\\
	\\
	It's important to note that $A[1] \leq A[2]$ and $A[n-1] \leq A[n]$. This implies that there $must$ be a peak somewhere between the first and last element. Even if $A[1]= A[n]$, by definition of a peak, $all$ of the elements between 1 and $n$ can be a peak if they're all the same number. \\
	\\
	Using these facts; I've come up with the following algorithm:\\
	\\
	$i =$ beginning of Array(or subarray)\\
	$j =$ end of Array (or subarray)\\
	findpeak $(A ,i, j)$ \\
	\indent let middle $= i + \frac{j-i}{2}$\\
	\\
	\indent if $A[middle-1] \leq A[middle]$ and $A[middle] \geq A[middle+1] \\
	\indent \rightarrow$ return $A[middle]$. \\
	\indent if $A[middle-1] > A[middle]$ (there must be a peak to the left)$ \\ 
	\indent \rightarrow$ return findpeak $(A , i, middle-1)$. \\
	\indent if $A[middle] < A[middle+1]$ (there must be a peak to the right)$ \\
	\indent \rightarrow$ return findpeak $(A ,middle+1, j)$. \\

	\begin{claim} 
	This algorithm will always terminate and return a peak.
\end{claim}

\begin{proof}
	Assume the algorithm did not return a peak. Then, there must not be an element $k \in \{2, 3, ..., n-1\}$ that is a peak, such that $A[k-1] \leq A[k]$ and $A[k] \geq A[k+1]$. \\
	\\ 
	Then, one of two cases must hold:\\
	Case 1: The array is constantly increasing, such that $\forall A[m]$, $m \in \{2, 3, ..., n\}$, $A[m-1] < A[m]$. But consider the case when $A[m]$ is the second to last element in the array, namely, $A[n-1]$. By definition, $A[n-1] \geq A[n]$, which is a contradiction. Then, $A[n-1]$ is a peak and it would be returned. Thus, it must be the case that $\exists A[m]$, $m \in \{1, 2, ... , n\}$ such that $A[m-1] \geq A[m]$.\\
	\\
	Case 2:  The array is constantly decreasing, such that $\forall A[m]$, $m \in \{2, 3, ..., n\}$, $A[m-1] > A[m]$. But consider the case when $A[m]$ is the second element in the array, namely, $A[2]$. By definition, $A[1] \leq A[2]$, which is a contradiction. Then, $A[2]$ is a peak, and it would be returned. Thus, it must be the case that $\exists A[m]$, $m \in \{1, 2, ... , n\}$ such that $A[m-1] \geq A[m]$.\\ 
	\\
	There are no other cases to consider, because all other cases would include a peak, so we can conclude that in all cases, a peak exists and will be returned.
	
\end{proof}

	\begin{claim} 
		This algorithm runs in $O (log$ $n)$ time.
	\end{claim}

	\begin{proof}
		Each recurrance, we are performing findpeak $(A ,i, j)$ on a subarray that is at most half the length of the original array, and are doing two comparisons.
		We get the following recurrence relation: \\
		$T(n) = T(\frac{n}{2}) + 2$ until $n = 1$. \\
		$\implies T(\frac{n}{2^k}) + 2k$\\
		$\implies 1 = \frac{n}{2^k}$\\
		$\implies 2^k = n$
		$\implies k = log_2 n$. 
		$\implies O(log$ $n)$
	\end{proof}

	\newpage

\end{document}

\title{Spherical Polyhedron}
\author{
	Alyssa H. Melton \\
	MATH 128A: Classical Geometry\\
}
\date{\today}

\documentclass[12pt]{article}

\begin{document}
	\maketitle
	
	\begin{abstract}
		We will explore properties and differences of tessellations in different geometries. In particular, looking at spherical geometry, tiling over a sphere, and the relation to polyhedra. 
	\end{abstract}
	
	\section{Tessellations and Tiling}
		A tessellation refers to a uniform tiling of a plane with polygons, such that an equal number of identical polygons meet at each vertex. We can denote tessellations in the following form: $\{a, b\}$ where $a$ is the number of sides each polygon has, and $b$ is the number of polygons that share a vertex. This is called a Schlafli symbol. 
		
	\paragraph{Euclidean Plane}
		There are precisely three possible tessellations of regular polygons in Euclidean Space, namely $\{3, 6\}$, 
		$\{6,3\}$ and $\{4,4\}$. We recognize these as the tiling of regular triangles, squares, and hexagons. 
	
	\paragraph{Hyperbolic Plane}
		The hyperbolic plane however has an infinite number of tessellations available for creation. We find that tessellations in hyperbolic space hold the following: $\frac{a}{a} + \frac{1}{b} < \frac{1}{2}$. This is true for infinitely many $a$ and $b$.  
	
	\paragraph{Surface of a Sphere}
		However, I am far more interested in tessellations over the surface of a sphere. Rather than speaking of tessellations, the tiling of a sphere is more commonly reffered to as spherical tiling, and the flattening of these tiles create spherical polyhedra. We say a spherical polyhedron is the partition of the surface of the sphere by \textit{great arcs}. Some examples that we recognize are the hosohedron, which resembles the a blow up beach ball, and the truncated icoshedron which resembles the common soccer ball. 
	
	\paragraph{Programming}
		While attempting to display these visually pleasing patterns, I ran into a lot of difficulty finding ways to recursively draw the same tiling pattern on a circle, disk, sphere or euclidean plane. In particular, attempting to plot tessellations in the Poincare Disk was difficult because of the hurdles I had to address with plotting lines. Because Processing takes cartesian coordinates, I had to first find a mapping of the upper half plane to a disk, and map the polar coordinates to cartesian coordinates. I found issues accounting for infinity and looping around the circle ensuring a clean match between the beginning of the recursive function and when it ties back to the beginning. 
		
	\paragraph{Conclusion}
		Unfortunately, instead of learning an extensive amount on one topic in geometry for this project, I learned a little about many things while testing the waters of a potential visual program idea, and running into barriers in implimentation. These topics include the Farey Ford Sequence / Circle Packing, Tessellations in hyperbolic space, tessellations in Euclidean Space, Spherical Geometry, Spherical Polyhedra, and regular polyhedra. Ultimately I found regular polyhedra and their construction incredibly interesting and wish I knew I'd wanted to pursue that topic from the beginning. 
	
\end{document}
\documentclass[11pt]{article}
\usepackage{fullpage,amsthm,amsfonts,amssymb,epsfig,amsmath,times,amsthm}
\usepackage{tabu}
\usepackage{pgfplots}

\newtheorem{theorem}{Theorem}
\newtheorem{claim}[theorem]{Claim}
\newcommand\tab{\setlength\parindent{24pt}}

\begin{document}

	\begin{center}
		{\bf\Large CMPS 102 --- Fall 2018 --  Homework 4}\\
		Alyssa Melton\\
		I have read and agree to the collaboration policy. \\
		Collaborators: none\\
	\end{center}

	%------------------------------------------------------------------------------
	%------------------------------------------------------------------------------

	\section*{Solution to Problem 4}
		Given a graph G and it's maximum flow $f$, when decreasing an edge $e^*$ by one, creating the new graph, H, we can find the new maximum flow $f^*$ by the following:\\
		\\
		If the capacity of $e^*$ is greater than the maximum flow $f$, then the edge is not fully saturated anyways, so it doesn't affect the flow and we may return $f$ as $f^*$.\\
		\\
		Otherwise, we need to make sure that the edge $e^*$ isn't a key edge when pushing maximum flow through the graph. We can do this by considering any path in $H$ that goes through $e^*$, and decreasing the forward capacity of the edges in that path by 1 and running a backward edge of value 1 through all the edges of the path as well. \\
		\\
		Then, we see if there is an augmenting path in the graph.
		If there is, then we know that the edge $e^*$ being decreased does not affect the total flow, and we can return $f$.
		If there is not, then we know that decreasing this edge interrupted flow within the graph, and we can return $f-1$.

	\begin{claim}
		This algorithm is $O(m+n)$ time.
	\end{claim}

	\begin{proof}
		To find a path from the source to t through the edge $e^*$, we can run BFS from $s$ to the first node in $e^*$ and from the second node in $e^*$ to $t$. BFS runs in $O(m+n)$. Finding an augmenting path is also done in $O(m+n)$, by BFS.
	\end{proof}

	\begin{claim}
	Proof of Correctness
\end{claim}

\begin{proof}
	If the capacity of $e^*$ is greater than $f$, we can push all of $f$ through $e^*$ and there would still be room left. This is trivial.

	When $e^*$ is less than or equal to $f$, and $e^*$ does not constrict the flow, decreasing all edges from $s$ through $e^*$ to $t$ creates augmented edges, because the flow routes elsewhere than through $e^*$.

	When $e^*$ is less than or equal to $f$, and $e^*$ does constrict the flow, decreasing all edges from $s$ through $e^*$ to $t$ does not create augmented edges, because the only direction the one unit of flow could have gone would be through $e^*$. 
\end{proof}

	\newpage


\end{document}
